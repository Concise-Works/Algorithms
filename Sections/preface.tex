\chapter*{Preface}

\begin{wrapfigure}{l}{0.3\textwidth}
    \includegraphics[width=0.3\textwidth]{./mebw.png}
\end{wrapfigure}

\textbf{Motivation:} I thought I was bad at math, and terrible at taking notes.
I passed all my classes through osmosis and forgot everything after the final exam; However, that wouldn't 
work at Boston University (BU) for Computer Science. I would go onto retake classes CS131 (Discrete Math), CS210 (Computer Systems),
CS235 (Algebraic Algorithms w/ Leonid Levin).

What changed was when I became a course assistant for CS112 (Introduction to Computer Science II). I had taken multiple 
Java classes before BU and was fairly comfortable. What scared me to death was the thought of not being able to answer a student's question.
This fear catapulted me into taking notes for the first time in a google doc. I would double study the material
so that I would be prepared for questions in office hours, and post the notes on piazza (online forum). The positive feedback 
reinforced my note-taking. Knowing that others relied on my notes pushed me to improve them.

Then once it came to other classes, I thought, why not make notes for them too? I would go onto learn LaTeX to make 
them look nice, and use \href{https://www.mathcha.io/editor}{mathcha.io}, to create diagrams. People began to use my 
notes more and more (often nagging me for when the next post was...). That pressure, combined with the pressure of passing,
and the love I had for explaining and drawing the material compounded into these notes.

I was so completely done with the mountains of stress. I would rather study everything correctly the first time. After that,
I never stressed for an exam again. I would actually never study, because the notes I took were a form of studying. I would always 
go to office hours and help others out. Once finals came around, I again, never studied, and got the best grades I've ever gotten in my life, 
and graduated feeling more accomplished than ever before.

Now after being through the thick of it, I truly believe anyone can be good at math, or any subject with the right amount of attention.
All advanced topics are just basic definitions stacked together. If one can master the basics, everything else becomes possible.\\
\textbf{Text-layout:} We start at high-level understandings (admittedly somewhat dry with definitions),
gaining familiarity with data-structures and algorithms (where the fun begins), number systems (e.g,. binary),
which will allow us to dive deeper and create our own theoretical models of computation, i.e., a 
computer (plenty visuals, less text). After such, we revisit algorithms and data-structures, learning the classic methods and strategies
that motivated our modern system today. \underline{This text covers BU \textbf{CS112}, \textbf{CS210}, and \textbf{CS330} focusing on theory.}

\clearpage
\newpage 

\emph{}

\begin{center}
    \vfill
    \Large{Big thanks to \textbf{Christine Papadakis-Kanaris}}\\
    \normalsize 
    for teaching Intro. to Computer Science II,\\
    \Large{\textbf{Dora Erdos}} and  \textbf{Adam Smith}\\
    \normalsize 
    for teaching BU CS330: Introduction to Analysis of Algorithms,\\
    with contributions from:\\
    \textbf{S. Raskhodnikova, E. Demaine, C. Leiserson, A. Smith, and K. Wayne},\\
    at Boston University
    

    \vspace{3em}
    \textcolor{red}{\textit{Please note:}} These are my personal notes, and while I strive for accuracy, there may be errors. I encourage you to always verify with your own class materials and other sources as well.\\
    Comments and suggestions for improvement are always welcome.
    \vfill
\end{center}

\noindent 
\textbf{Support:} If you want to support, give a star on github (\url{https://github.com/Concise-Works/Algorithms}) or share with your friends!\\
\textbf{Contribute:} If you find an error, want to add something, or revise anything, consider forking and contributing, via the 
above github link. Pull requests are welcome! Submit practice problems, engage with the text, take ownership of your knowledge.\\
\textbf{Code Examples:} Many of the algorithms in this text are implemented here: \url{https://github.com/Concise-Works/Algorithms/blob/main/Implementations/alg.py}.
\textbf{All Illustrations:} Free for use and are accessible here (\href{https://www.mathcha.io/editor/OXK57SpXSQqTKBP9xGiQ84MZkhgM3XOeCJQ6z12}{matcha.io - 330})



\newpage 

\noindent
\textbf{Contributors (GitHub):}

\newcommand{\contrib}[2]{%
    \begin{minipage}[c]{\linewidth}
        \begin{tikzpicture}[baseline={(0,0)}]
             \clip (0,0) circle (12pt);
             \node[anchor=center] at (0,0) {\includegraphics[width=24pt]{Sections/contributors/#1.png}};
        \end{tikzpicture}
        \hspace{2mm}
        \href{https://github.com/#1}{\textbf{#2}}
    \end{minipage}
}

\vspace{1em}

\begin{itemize}
    \item \contrib{neezacoto}{Christian Rudder}
    \item \contrib{jakegustin}{Jake Gustin} 
    \item \contrib{onkr0d}{Ivan Khramtchenko}
    \item \contrib{Fadeleke57}{Farouk Adeleke}
\end{itemize}