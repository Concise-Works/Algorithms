\chapter*{Preface}

\noindent
We start at high-level understandings (admittedly somewhat dry with definitions),
gaining familiarity with data-structures and algorithms (where the fun begins), number systems (e.g,. binary),
which will allow us to dive deeper and create our own theoretical models of computation, i.e., a 
computer (plenty visuals, less text). After such, we revisit algorithms and data-structures, learning the classic methods and strategies
that motivated our modern system today (The must knows for any interview or real-world application).

\begin{center}
    \vfill
    \Large{Big thanks to \textbf{Christine Papadakis-Kanaris}}\\
    \normalsize 
    for teaching Intro. to Computer Science II,\\
    \Large{\textbf{Dora Erdos}} and  \textbf{Adam Smith}\\
    \normalsize 
    for teaching BU CS330: Introduction to Analysis of Algorithms,\\
    with contributions from:\\
    \textbf{S. Raskhodnikova, E. Demaine, C. Leiserson, A. Smith, and K. Wayne},\\
    at Boston University
    

    \vspace{3em}
    \textcolor{red}{\textit{Please note:}} These are my personal notes, and while I strive for accuracy, there may be errors. I encourage you to refer to the original slides for precise information.\\
    Comments and suggestions for improvement are always welcome.
    \vfill
\end{center}
