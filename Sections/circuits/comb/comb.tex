\newpage
\subsection{Combinational Logic}

Truth tables quickly grow as $n$ inputs lead to $2^n$ rows in the truth table. So instead, 
we model functions using boolean algebra (e.g., $F = A \cdot B + \overline{C}$), and then implement them using logic gates.

\begin{figure}[ht!]
  \centering
  \includegraphics[width=.9\textwidth]{Sections/circuits/comb/comb.png}
  \caption{1) Shows the desired circuit function, 2) the corresponding truth table, and 3) the boolean logic. From the 
  truth table, we can pick a row, substituting the inputs $A$, $B$, and $C$ into the boolean expression, we expect to see the output corresponding to $F$.
  For example, row 3, $F=\overline{C}\overline{B}A+\overline{C}BA+CB\overline{A}+CBA$, is, $0 = (\overline{0}\cdot 1 \cdot 0) + (\overline{0} \cdot 1 \cdot 0) + (0 \cdot 1 \cdot 0) + (0 \cdot 1 \cdot 0).$}
  \label{fig:comb-logic}
\end{figure}

\begin{Def}[Logic Gates]

    Below are the common logic gates with the addition of the resistor which limits current:
    \begin{center}
        \includegraphics[width=.8\textwidth]{Sections/circuits/comb/logic_gates.png}
    \end{center}

    \noindent
    The resistor has many symbols; However, we focus purely on logic gates in this text, and how we can make 
    more complex logical systems from these basic building blocks.
\end{Def}

\newpage 

\noindent
For example, lets consider just NOT, AND, and OR gates and their truth tables:

\begin{figure}[ht!]
  \centering
  \includegraphics[width=.9\textwidth]{Sections/circuits/comb/logic_gates_2.png}
  \caption{1) NOT gate 2) AND gate 3) OR gate. Recall that inside these gates is a combination of PUNs and PDNs to achieve the desired logic, which here we have abstracted away.}
  \label{fig:nand-nor}
\end{figure}

\begin{figure}[ht!]
  \centering
  \includegraphics[width=1\textwidth]{Sections/circuits/comb/logic_gates_3.png}
  \caption{Modeling $F=\overline{C}\overline{B}A+\overline{C}BA+CB\overline{A}+CBA$ with logic gates. Note, 
  we see that the ANDS and ORS have more than 2 inputs; They operate the same way as their 2-input counterparts, just with more inputs.}
  \label{fig:logic-gates-model}
\end{figure}

\newpage 

\begin{theo}[Increasing Gate Input Size]

    Above in Figure \ref{fig:logic-gates-model}, we see AND and OR gates with more than 2 inputs. There are 
    multiple ways to implement these larger gates, observe below:\\

    \noindent
    \includegraphics[width=1\textwidth]{Sections/circuits/comb/increase_gate.png}\\

    \noindent
    1) is a 3 input gate, and 2) is a 4 input gate; This method of attaching the gates in sequential order is called
    \textbf{chaining}. 3) takes a different approach and instead uses a \textbf{tree method}.\\

    \noindent
    \rule{\textwidth}{0.4pt}\\
    
    \noindent
    In terms of $t_{PD}$, assuming all gates have the same delay, \underline{\textbf{chaining grows linearly}} with the number of inputs, and
    the \underline{\textbf{tree method grows logarithmically}} with the number of inputs.

    \textbf{However}, if the gates were to have \textbf{different delays}, introduces \textbf{bottlenecks}; In the case of the tree method, getting 
    an input like $D$ introduces a longer path, while in the sequential method, $D$ has less impact on the overall delay.
    \end{theo}

\begin{theo}[NAND AND NOR - Gate Increase Problem \& Efficiency]

    \underline{NAND and NOR are \textbf{not associative} like AND and OR.} This means that we cannot use the chaining or tree method
    to increase the number of inputs for NAND and NOR gates.\\

    \noindent
    Additionally, NAND and NOR gates are more efficient in CMOS logic, as CMOS is naturally inverting with PUNs and PDNs. Thus, their implementations
    are simpler than AND and OR gates.
\end{theo}

\noindent
To further illustrate that NAND and NOR are functionally complete, observe the below diagram:

\begin{figure}[ht!]
  \centering
  \includegraphics[width=1.1\textwidth]{Sections/circuits/comb/nand_nor_fp.png}
  \caption{ This diagram shows that NOT, AND, and OR gates can be constructed purely from NAND (left) or NOR (right) gates.}
  \label{fig:nand-nor-complete}
\end{figure}

\begin{theo}[Increasing NAND Gate Input Size with 2-input NANDs]

    \label{theo:nand-increase}

    Start with the desired function, e.g., $\overline{A \cdot B \cdot C}$, for clarity $\texttt{NAND}(A,B,C)$.
    We attempt to group inputs into pairs of 2, using identities from Figure (\ref{fig:nand-nor-complete}): $\texttt{NAND}(A,B,C) = \texttt{NOT}(\texttt{AND}(A,B,C))$.
    Now we group:
    $\texttt{NOT}(\texttt{AND}(A,B,C)) = \texttt{NOT}(\texttt{AND}(\texttt{AND}(A,B),C))$, and we are done:\\

    \noindent
    \includegraphics[width=1\textwidth]{Sections/circuits/comb/increase_gate2.png}\\
    \noindent
    To only use NAND gates, we can start simplifying by bringing the NOT and AND back together:
    $\texttt{NAND}(\texttt{AND}(A,B),C)$. Though we still have an AND gate, we can replace it with NANDs using Figure \ref{fig:nand-nor-complete}:\\
    
    \noindent
    \includegraphics[width=1\textwidth]{Sections/circuits/comb/increase_gate3.png}

\end{theo}


\begin{Note}
\textbf{Note:} As we've mentioned before,
chaining vs. tree methods vary differently on $t_{PD}$. Additionally notice that the size of our circuit can fluctuate based
on how we decide to break down our boolean expression. In the above Theorem (\ref{theo:nand-increase}), if we assume
AND gates are slower than NAND gates in this system, the simplified NAND version is faster, and smaller, assuming AND gates are composed of multiple NANDs.
\end{Note}