\subsection{Sequential Logic: Building Memory (RAM/DRAM)}

\noindent
If we want to have a circuit that can \textbf{store} information, say ``Do $x$ if the previous input was $y$''
(E.g., traffic lights).

To start building intuition lets start by seeing what happens when we connect
gate outputs to other gate inputs in a loop, creating a \textbf{feedback loop} \cite{core_dumped_transistors}:
\begin{figure}[ht!]
  \centering
  \includegraphics[width=1\textwidth]{Sections/circuits/state/or_loop.png}
  \caption{A simple feedback loop using OR gates. 1) Initially both inputs are zero, then the free input
  is set to 1. 2) The output becomes 1, switching the feedback input to 1. 3) 
  Now even if the free input is set back to 0, the output remains 1, since one of the OR inputs is still 1.}
  \label{fig:or_loop}
\end{figure}

\begin{figure}[ht!]
  \centering
  \includegraphics[width=1\textwidth]{Sections/circuits/state/and_loop.png}
  \caption{A simple feedback loop using AND gates. 1) Initially both inputs are one, then the free input
  is set to 0. 2) The output becomes 0, switching the feedback input to 0. 3) 
  Now even if the free input is set back to 1, the output remains 0, since one of the AND inputs is still 0.}
  \label{fig:and_loop}
\end{figure}

\begin{figure}[ht!]
  \centering
  \includegraphics[width=1\textwidth]{Sections/circuits/state/combine_loop.png}
  \caption{Combining the OR and AND feedback loops from Figures (\ref{fig:or_loop}) and (\ref{fig:and_loop}),
  we get the above. 1) Initially both inputs are 0, output 0, then inputs both are set to 1, resulting in a a 1 output. 2) The output is now 1, and the first 
  input is set to 0, but the output remains 1, as the second input is still 1, driving the AND gate.}
  \label{fig:combine_loop}
\end{figure}

\newpage 
\noindent
In the above Figure (\ref{fig:combine_loop}), we can see that turning on the second input, effectively resets 
the output to 0. Lets see what happens when we keep the second input on with an inverter:

\begin{figure}[ht!]
  \centering
  \includegraphics[width=1\textwidth]{Sections/circuits/state/and_loop2.png}
  \caption{1) Initially all inputs are 0, output 0. 2) First input is set to 1,
  output becomes 1. 3) First input is set back to 0, but output remains 1. 
  4) Second input is set to 1, output becomes 0. 5) We characterize the first 
  input as the `SET' and the second input as the `RESET'.}
  \label{fig:sr_latch}
\end{figure}

\noindent
This circuit is called an:
\begin{Def}[AND-OR Latch]

    An \textbf{AND-OR Latch} is a basic memory element that can store one bit of information. 
    It has two inputs, labeled `S' (Set) and `R' (Reset), and a single output `Q'.

    However, this design has a critical flaw: if both `S' and `R' are set to 1 simultaneously creates an 
    \textbf{invalid state}, and the output hangs on 0, ignoring the 1 that's being ``set''.
\end{Def}

\noindent
Next we create a more sophisticated latch that avoids this invalid state.

\newpage 