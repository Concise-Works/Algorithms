\begin{theo}[Common Derivatives]

    \begin{tabular}{ll}
        Power Rule: For \( n \neq 0 \) & \( \frac{d}{dx}(x^n) = n \cdot x^{n-1} \) . E.g., \( \frac{d}{dx}(x^2) = 2x \) \\
        Derivative of a Constant: & \( \frac{d}{dx}(c) = 0 \) . E.g., \( \frac{d}{dx}(5) = 0 \) \\
        Derivative of \( \ln \): & \( \frac{d}{dx}(\ln x) = \frac{1}{x} \) \\
        Derivative of \( \log_a \): & \( \frac{d}{dx}(\log_a x) = \frac{1}{x \ln a} \) \\
        Derivative of \( \sqrt{x} \): & \( \frac{d}{dx}(\sqrt{x}) = \frac{1}{2\sqrt{x}} \) \\
        Derivative of function \( f(x) \): & \( \frac{d}{dx}(x) = 1 \) . E.g., \( \frac{d}{dx}(5x) = 5 \) \\
        Derivative of the Exponential Function: & \( \frac{d}{dx}(e^x) = e^x \) \\
    \end{tabular}
\end{theo}

\begin{theo}[L'Hopital's Rule]

    \label{thm:lhospital}
    
    Let \(f(x)\) and \(g(x)\) be two functions. If \(\lim_{x\to a}f(x) = 0\) and \(\lim_{x\to a}g(x) = 0\), or\\ \(\lim_{x\to a}f(x) = \pm\infty\) and \(\lim_{x\to a}g(x) = \pm\infty\), then:
    \[\lim_{x\to a}\dfrac{f(x)}{g(x)} = \lim_{x\to a}\dfrac{f'(x)}{g'(x)}\]
    Where \(f'(x)\) and \(g'(x)\) are the derivatives of \(f(x)\) and \(g(x)\) respectively.
\end{theo}

\newpage 

\begin{theo}[Exponents Rules]

    \label{thm:addexp}
    
    For $a,b,x\in\mathbb{R}$, we have:
    \Large
    \[x^a\cdot x^b = x^{a+b} \text{ and } (x^a)^b=x^{ab}\]
    \[x^a\cdot y^a = (xy)^a \text{ and } \dfrac{x^a}{y^a} = \left(\dfrac{x}{y}\right)^a\]
    \normalsize

\end{theo}

\begin{Note}
    \textbf{Note:} The $:=$ symbol is short for ``is defined as.'' For example, $x:=y$ means $x$ is defined as $y$.
\end{Note}

\begin{Def}[Logarithm]

    Let $a,x\in\mathbb{R}$, $a>0$, $a\neq 1$. Logarithm $x$ base $a$ is denoted as $\log_a(x)$, and is defined as:
    \Large
    \[\log_a(x)=y\Longleftrightarrow a^y=x\]
    \normalsize
    Meaning $log$ is inverse of the exponential function, i.e., $\log_a(x):=(a^y)^{-1}$.
\end{Def}
\begin{Tip}
    To remember the order $log_a(x)=a^y$, think, ``base $a$,'' as $a$ is the base of our $log$ and $y$.
\end{Tip}

\begin{theo}[Logarithm Rules]

    \label{thm:logrules}
    
    For $a,b,x\in\mathbb{R}$, we have:
    \Large
    \[\log_a(x)+\log_a(y)=\log_a(xy) \text{ and } \log_a(x)-\log_a(y)=\log_a\left(\dfrac{x}{y}\right)\]
    \[\log_a(x^b)=b\log_a(x) \text{ and } \log_a(x)=\dfrac{\log_b(x)}{\log_b(a)}\]
    \normalsize
\end{theo}

\newpage

\begin{Def}[Permutations]
    
        Let $n\in\mathbb{Z^+}$. Then the number of distinct ways to arrange $n$ objects in order is\\
        $n!:=n\cdot(n-1)\cdot(n-2)\cdot\ldots\cdot2\cdot1$.
         When we choose $r$ objects from $n$ objects, it's Denoted:
        \Large
        \[^nP_r:=\dfrac{n!}{(n-r)!}\]
        \normalsize
        Where $P(n,r)$ is read as ``$n$ permute $r$.''
\end{Def}

\begin{Def}[Combinations]

    Let $n$ and $k$ be positive integers. Where order doesn't matter, the number of distinct ways to choose $k$ objects from $n$ objects is it's \textit{combination}. Denoted:
    \Large
    \[\binom{n}{k}:=\dfrac{n!}{k!(n-k)!}\]
    \normalsize

    \noindent
    Where $\binom{n}{k}$ is read as ``$n$ choose $k$.'', and $\binom{\cdot}{\cdot}$, the \textit{binomial coefficient}.
    
\end{Def}

\begin{theo}[Binomial Theorem]

    \label{thm:binomial}
    
    Let \(a\) and \(b\) be real numbers, and \(n\) a non-negative integer. The binomial expansion of \((a + b)^n\) is given by:
    
    \[
    (a + b)^n = \sum_{k=0}^{n} \binom{n}{k} a^{n-k} b^k
    \]
    which expands explicitly as:
    
    \[
    (a + b)^n = \binom{n}{0} a^n + \binom{n}{1} a^{n-1}b + \binom{n}{2} a^{n-2}b^2 + \cdots + \binom{n}{n-1} a b^{n-1} + \binom{n}{n} b^n
    \]
    \noindent
    where \(\binom{n}{k}\) represents the binomial coefficient, defined as:
    \Large
    \[
    \binom{n}{k} = \frac{n!}{k!(n-k)!}
    \]
    \normalsize
    for \(0 \leq k \leq n\).

\end{theo}

\newpage

\begin{theo}[Binomial Expansion of \(2^n\)]

    \label{thm:binomial_2n}

    For any non-negative integer \(n\), the following identity holds:
    
    \[
    2^n = \sum_{i=0}^{n} \binom{n}{i} = (1 + 1)^n.
    \]

\end{theo}

\begin{Def}[Well-Ordering Principle]

    \label{def:well_ordering_principle}

    Every non-empty set of positive integers has a least element.
\end{Def}

\begin{Def}[``Without Loss of Generality'']

    \label{def:wlog}

    A phrase that indicates that the proceeding logic also applies to the other cases.
    i.e., For a proposition not to lose the assumption that it works other ways as well.

\end{Def}
\begin{theo}[Pigeon Hole Principle]

    Let \( n, m \in \mathbb{Z}^+ \) with \( n < m \). Then if we distribute \( m \) pigeons into \( n \) pigeonholes, there must be at least one pigeonhole with more than one pigeon.
\end{theo}

\begin{theo}[Growth Rate Comparisons]

    \label{thm:growth_rates}

    Let $n$ be a positive integer. The following inequalities show the growth rate of some common functions in increasing order:
    \LARGE
    \[
    1 < \log n < n < n \log n < n^2 < n^3 < 2^n < n!
    \]
    \normalsize
    These inequalities indicate that as $n$ grows larger, each function on the right-hand side grows faster than the ones to its left.
    
\end{theo}






